\documentclass{article}

\usepackage{lipsum}
\usepackage[margin=2cm, left=2cm, includefoot]{geometry}
\usepackage{graphicx}
\usepackage{float}
\usepackage{hyperref}

% Header and footer
\usepackage{fancyhdr}
\pagestyle{fancy}

\rhead{}
\lhead{}
\fancyfoot{}
\fancyfoot[R]{\thepage}
\renewcommand{\headrulewidth}{0pt}
\renewcommand{\footrulewidth}{0pt}
%

\begin{document}

\begin{titlepage}
	\begin{center}
		\line(1,0){500}\\
		[6mm]
		\huge{\bfseries Testing Document}\\
		\line(1,0){500}\\
		[5mm]
		\includegraphics[width=150px]{../images/AWorldOfPlants.png}
		\\
		[5mm]
		\large\textbf{Project:}\\A World of Things\\
		[3mm]
		\large\textbf{Client:}\\Julian Hambleton-Jones\\
		[3mm]
		\large \textbf{Team:}\\Funge\\
		\line(1,0){500}\\
		[5mm]
		\large \textbf{Team Members:}\\
		[3mm]
		\large 14214742 - Matthew Botha\\
		\large 14446619 - Gian Paolo Buffo\\
		\large 14027021 - Matthias Harvey\\
        \large 14035538 - Dillon Heins\\[3mm]
	\end{center}
\end{titlepage}

\cleardoublepage
\thispagestyle{empty}
\tableofcontents
\cleardoublepage
\setcounter{page}{1}

\textit{Disclaimer: The system is still under development. All documentation will be updated as the system progresses}

\section{Introduction}
	In order to test our system, we need to do unit tests, integration and system tests and performance and stress tests. There are two main halves, namely the back end and front end.

\section{IoT Back End}
	\subsection{Objectives and Tasks}
		In order to fully test the IoT back end component of the project, we need to ensure that a number of connections are in place and are working as expected:
		\begin{itemize}
			\item{From the device to IoT}
			\item{From IoT to Lambda}
			\item{From Lambda to DynamoDB}
			\\
			\item{From API Gateway to Lambda}
			\item{From Lambda to the device's shadow via IoT}
			\item{From the shadow to the device}
		\end{itemize}
	
	\subsection{Scope}
		For each of the above mentioned steps, we must write a unit test that is automated. We must then write tests to test
		the integration of each of the steps with each other. Finally, we must test how the back end functions with the front
		end, testing the system as a whole.
	
	\subsection{Testing Strategy}
		\subsubsection{Unit Testing}
			In order to test the functionality of the device, we have created a mock device using the AWS IoT Device SDK for
			JavaScript, and have coded the mock device using NodeJS.
			\\\\
			\textbf{IoT rule test}:
			\\
			When the mock device test is run, it will send an MQTT message to the IoT platform. From here, we can check whether
			the device shadow has been updated and if the rule has been triggered. If the shadow is updated, then the connection
			was successful. If the connection was successful but the rule was not triggered, then the rule is not working. We
			still need to automate this process.
			\\\\
			\textbf{Lambda Function Test}:
			\\
			The AWS Lambda console has testing integrated into the platform. We have written a test package to send to Lambda
			that simulates the messages sent from an IoT device through an IoT rule. The test runs a mock DynamoDB database
			and will return if the test was successful or not. We can also use the logging feature to check the test results.
			We still need to automate this process.
			\\\\
			\textbf{API Gateway and Thing Shadows}:
			\\
			Because we have not yet implemented the thing shadow interface and functionality, there is no testing for the API
			Gateway connection or thing shadow yet.
			
			\subsubsection{System and Integration Testing}
			When the mock device test is run, it will send a message through the IoT back end system and leave an item in the 
			DynamoDB tables. If the test is successful, then we can see the messages in the DynamoDB table. We still have to 
			automate this process.
			
			\subsubsection{Performance and Stress Testing}
			There is no testing for performance or stress testing as of yet.
\newpage

\section{Front End}
	\subsection{Objectives and Tasks}
		
	\subsection{Scope}
		
	\subsection{Testing Strategy}
		\subsubsection{Unit Testing}
		\subsubsection{System and Integration Testing}
		\subsubsection{Performance and Stress Testing}

\end{document}