\documentclass{article}

\usepackage[margin=2cm, left=2cm, includefoot]{geometry}
\usepackage{graphicx}
\usepackage{float}
\usepackage{hyperref}

% Paragraph formatting
\usepackage{indentfirst}
\setlength{\parindent}{4em}
\setlength{\parskip}{1em}

% Header and footer
\usepackage{fancyhdr}
\pagestyle{fancy}

\rhead{}
\lhead{}
\fancyfoot{}
\fancyfoot[R]{\thepage}
\renewcommand{\headrulewidth}{0pt}
\renewcommand{\footrulewidth}{0pt}
%

\begin{document}

\begin{titlepage}
	\begin{center}
		\line(1,0){500}\\
		[6mm]
		\huge{\bfseries Architectural and Functional Requirements}\\
		\line(1,0){500}\\
			[5mm]
			\includegraphics[width=150px]{images/AWorldOfPlants.png}
			\\
		[5mm]
		\large\textbf{Project:}\\A World of Things\\
		[3mm]
		\large\textbf{Client:}\\Julian Hambleton-Jones\\
		[3mm]
		\large \textbf{Team:}\\Funge\\
		\line(1,0){500}\\
		[5mm]
		\large \textbf{Team Members:}\\
		[3mm]
		\large 14214742 - Matthew Botha\\
		\large 14446619 - Gian Paolo Buffo\\
		\large 14027021 - Matthias Harvey\\
        \large 14035538 - Dillon Heins\\[3mm]
	\end{center}
\end{titlepage}

\cleardoublepage
\thispagestyle{empty}
\tableofcontents
\begin{table}[]
\centering
\caption{Version Table}
\label{my-label}
\begin{tabular}{lllll}
\cline{1-3}
\multicolumn{1}{|l|}{\textbf{Version}} & \multicolumn{1}{l|}{\textbf{Date}} & \multicolumn{1}{l|}{\textbf{Description}}                                                                                                                                                                                                            &  &  \\ \cline{1-3}
\multicolumn{1}{|l|}{0.1}              & \multicolumn{1}{l|}{22/05/2016}    & \multicolumn{1}{l|}{Vision, scope, architectural requirements and initial architecture design.}                                                                                                                                                      &  &  \\ \cline{1-3}
\multicolumn{1}{|l|}{0.2}              & \multicolumn{1}{l|}{29/07/2016}    & \multicolumn{1}{l|}{\begin{tabular}[c]{@{}l@{}}Creation of separate documents for architecture design, software\\ requirements, testing and user manual. Each populated with the relevant\\ information for the project at this stage.\end{tabular}} &  &  \\ \cline{1-3}
                                       &                                    &                                                                                                                                                                                                                                                      &  & 
\end{tabular}
\end{table}
\cleardoublepage
\setcounter{page}{1}

\section{Introduction}

\section{Vision}

\section{Background}

\subsection{Future business/research opportunities}

\subsection{The Client's Problem}

\section{Important Technology}

\section{Architecture Requirements}

\subsection{Access Channel Requirements}

\subsubsection{Human Access Channels}

\subsubsection{System Access Channels}

	\subsection{Quality Requirements}
		\subsubsection{Performance}
			In order to have our system be as user friendly as possible, it is important to have the user interface respond quickly with little delay. The website should load quickly and logging in and out should be as fast as possible. The response to the live feed of MQTT topic updates in the live view should be instantaneous in order to allow for an accurate live graphing of the data.
				
			On the server side, Lambda function executions and database read/writes should be speedy to ensure that bottlenecks do not occur. The response to any MQTT messages received should be instantaneous in order to process multiple MQTT messages properly.
			
		\subsubsection{Usability}
			Because our system might be used by persons not familiar with complex systems, the user interface should be intuitive and efficient to use. Users with basic computer literacy should be able to easily use the core functionality of the system with little to no extra training required. Beyond the user interface, the user manual should be clear and well laid out in order to assist with learning of the system. To further ease of use, input validation should be done on the client side as much as possible to ensure that validation is fast and does not require a server response.
			
			Error messages generated by the system while in use should be clear and informative, helping the user understand and fix any issues that may have occurred. The messages should not be too technical in nature and attempt to provide a solution to the error.
		\subsubsection{Reliability}
			The system should be constantly available to the user via the web portal. This means that the server should not be down at any point. In order to ensure this, there should not be a single point of failure in the system.
			
			The devices send MQTT messages on a Quality of Service level of zero (QoS 0). This means that if a message is sent but not received, it is not resent. This means that the server should always be connected and waiting for MQTT messages.
		\subsubsection{Scalability}
			
			\begin{itemize}
				\item System will be able to support 10 000 concurrent users and should be able to scale upwards to handle hundreds of thousands of concurrent calls
				\item The system will be able to support up to 2000 plant boxes and should be able to scale proportionally with regard to the number of users (Up to hundreds of thousands of concurrent calls)
			\end{itemize}
		\subsubsection{Flexibility}
			It is important that the system has support for many hardware components such as sensors. These technologies are advancing rapidly, therefore it should be easy to add new components to the system, without making any large changes.
			\begin{itemize}
				\item Flexibility in terms of easily adding or removing resources and functions to API Gateway
				\item Easily add or remove Lambda functions
				\item The system should be modular in the sense that it is able to function with any number of sensors. It is not imperative that a user has all the necessary sensors in order to use the system
			\end{itemize}
		\subsubsection{Security}
			\begin{itemize}
				\item All personal user information should be kept private and should not be accessible by other users
				\item User authentication and authorization should be performed
				\item Passwords should be hashed and salted
				\item Personal information such as passwords should be encrypted when being sent over the network
			\end{itemize}
		\subsubsection{Auditability/Monitorability}
			\begin{itemize}
				\item API Gateway should monitor performance metrics as well as information on API calls, data latency and error rates.
				\item Conditions of the environment of the plants should be monitored in real time
				\item Each sensor should be distinguishable from the others so as to track where the data originates from
				\item The logs of the sensors will be persisted to the database so that analytics can be performed on them
				\item The results of the sensor logs will be made available via the user interface to the users
				\item The system will not allow the audit log to be modified
			\end{itemize}
		\subsubsection{Integrability}
			\begin{itemize}
				\item The system should be able to integrate well between the Amazon Web Services which are being used
				\item Future integration requirements should be easily addressable by using widely adopted public standards
				\item The system should be able to integrate with different hardware as long as the hardware queues information using a standardized format using MQTT
			\end{itemize}
	\subsection{Integration Requirements}
	\subsection{Architecture Constraints}

\section{Architecture Design}

\subsection{Infrastructure}

\subsection{Services}

\subsubsection{TODO add services here}

\subsection{Tactics}

\subsubsection{Flexibility}

\subsubsection{Reliability}

\subsubsection{Security}

\subsubsection{Auditability}

\section{Database and Persistence}

\section{Process Specification}

\subsection{Server}

\subsection{Frontend}

\section{Functional Requirements}

\subsection{Use Case Prioritisation}

\subsection{Use Case/Service Contracts}

\subsubsection{Add use cases/service contracts}

\section{Technologies}

\subsection{Todo add technologies like framework, web container etc}

\section{Initial Design}

\section{Open Issues}

\end{document}
