\documentclass{article}

\usepackage{lipsum}
\usepackage[margin=2cm, left=2cm, includefoot]{geometry}
\usepackage{graphicx}
\usepackage{float}
\usepackage{hyperref}
\usepackage{color}

% Header and footer
\usepackage{fancyhdr}
\pagestyle{fancy}

\rhead{}
\lhead{}
\fancyfoot{}
\fancyfoot[R]{\thepage}
\renewcommand{\headrulewidth}{0pt}
\renewcommand{\footrulewidth}{0pt}
%

\begin{document}

\begin{titlepage}
	\begin{center}
		\line(1,0){500}\\
		[6mm]
		\huge{\bfseries User Manual}\\
		\line(1,0){500}\\
		[5mm]
		\includegraphics[width=150px]{../images/AWorldOfPlants.png}
		\\
		[5mm]
		\large\textbf{Project:}\\A World of Things\\
		[3mm]
		\large\textbf{Client:}\\Julian Hambleton-Jones\\
		[3mm]
		\large \textbf{Team:}\\Funge\\
		\line(1,0){500}\\
		[5mm]
		\large \textbf{Team Members:}\\
		[3mm]
		\large 14214742 - Matthew Botha\\
		\large 14446619 - Gian Paolo Buffo\\
		\large 14027021 - Matthias Harvey\\
        \large 14035538 - Dillon Heins\\[3mm]
	\end{center}
\end{titlepage}

\cleardoublepage
\thispagestyle{empty}
\tableofcontents
\cleardoublepage
\setcounter{page}{1}

\textit{Disclaimer: The system is still under development. All documentation will be updated as the system progresses}

\section{Introduction}
	This user manual will help you familiarise yourself with the ins and outs of the User Interface, from registration and login to creating, viewing and managing your own growing plants.
	A World of Plants has been designed to make independent agriculture a fun and personalised experience. Compete to unlock new rewards or progress at your own leisure. Follow the plant suggestions to-the-letter or follow your own intuition. It's all up to you.
	Welcome to A World of Plants. 
	
\section{Landing Page}
	\subsection{Logging in}
		\textcolor{red}{(1)} The first page you'll encounter is the login page. If you already have an account, enter your username and password in the appropriate blocks and click "Sign In".
		\newline
		\textcolor{red}{(2)} If you'd like to find out more about A World of Plants, you can scroll down to our information page. Scroll back up when you're ready to login or register.
		\newline
		\textcolor{red}{(3)} If you have not yet registered, click on "Need an account?" to navigate to the account registration.
		\begin{figure}[H]
			\includegraphics[width=\textwidth]{../images/UserManual/login.PNG}
			\caption{Login Page}
		\end{figure}				
		
	\subsection{Account Registration}
		\textcolor{red}{(1)} If you do not yet have an account, enter your email, the username you'd like to be known by, and your password (twice) and click the "Sign Up" button.
		\newline
		\textcolor{red}{(2)} If you already have an account and would like to log in, click on "Have an Account?" to navigate to the user login.
		\newline
		\begin{figure}[H]
			\includegraphics[width=\textwidth]{../images/UserManual/registration.PNG}
			\caption{Registration Page}
		\end{figure}
		
\section{Main Site}
	\subsection{Navbar}
	All pages on the main site will display the navbar at the top.
		\subsubsection{User Actions}
		\textcolor{red}{(1)} To access your user actions, click on your username at the right of the navbar. From here, you can either navigate to your profile or log out.
		\begin{figure}[H]
			\includegraphics[width=\textwidth]{../images/UserManual/navbar-user-actions.PNG}
			\caption{Navbar User Actions}
		\end{figure}
		
	\subsection{Plants}
	To access your plants, click on "Plants" in the sidebar. This page will display a summary of all your current plants.
	\newline
	\textcolor{red}{(1)} To add a new plant, click on the "Create New Plant" button on the top right. See \hyperref[sec:creating-a-plant]{Creating a Plant} for more information.
	\newline
	\textcolor{red}{(2)} To view more information or change some of the details for a specific plant, click on "View Details" for the plant of your choice. See \hyperref[sec:plant-details]{Plant Details} for more information.
	\begin{figure}[H]
		\includegraphics[width=\textwidth]{../images/UserManual/plants.PNG}
		\caption{Plants View}
	\end{figure}
		
		\subsubsection{Creating a Plant}
		\label{sec:creating-a-plant}
		To create a plant, enter the plant's unique details and then click on "Create".
		\begin{figure}[H]
			\includegraphics[width=\textwidth]{../images/UserManual/create-plant.PNG}
			\caption{Creating a Plant}
		\end{figure}
		
	\subsection{Plant Details}
	\label{sec:plant-details}
	From here you can get an overview of all of your plant's details.
	\newline
	\textcolor{red}{(1)} To update plant details such as the plant's name or category, make your desired changes and then click the "Update" button.
	\newline
	\textcolor{red}{(2)} If you'd like to delete your plant, click on the red "Delete" button. \textbf{This is permanent}, so be careful.
	\begin{figure}[H]
		\includegraphics[width=\textwidth]{../images/UserManual/plant-details.PNG}
		\caption{Plant Details}
	\end{figure}

\end{document}