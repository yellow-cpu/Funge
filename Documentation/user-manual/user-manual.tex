\documentclass{article}

\usepackage{lipsum}
\usepackage[margin=2cm, left=2cm, includefoot]{geometry}
\usepackage{graphicx}
\usepackage{float}
\usepackage{indentfirst}
\usepackage{listings}
\usepackage{hyperref}
\usepackage{color}

% Header and footer
\usepackage{fancyhdr}
\pagestyle{fancy}

\rhead{}
\lhead{}
\fancyfoot{}
\fancyfoot[R]{\thepage}
\renewcommand{\headrulewidth}{0pt}
\renewcommand{\footrulewidth}{0pt}
%

\begin{document}

\begin{titlepage}
	\begin{center}
		\line(1,0){500}\\
		[6mm]
		\huge{\bfseries User Manual}\\
		\line(1,0){500}\\
		[5mm]
		\includegraphics[width=150px]{../images/AWorldOfPlants.png}
		\\
		[5mm]
		\large\textbf{Project:}\\A World of Things\\
		[3mm]
		\large\textbf{Client:}\\Julian Hambleton-Jones\\
		[3mm]
		\large \textbf{Team:}\\Funge\\
		\line(1,0){500}\\
		[5mm]
		\large \textbf{Team Members:}\\
		[3mm]
		\large 14214742 - Matthew Botha\\
		\large 14446619 - Gian Paolo Buffo\\
		\large 14027021 - Matthias Harvey\\
        \large 14035538 - Dillon Heins\\[3mm]
	\end{center}
\end{titlepage}

\cleardoublepage
\thispagestyle{empty}
\tableofcontents
\begin{table}[]
\centering
\caption{Version Table}
\label{my-label}
\begin{tabular}{lllll}
\cline{1-3}
\multicolumn{1}{|l|}{\textbf{Version}} & \multicolumn{1}{l|}{\textbf{Date}} & \multicolumn{1}{l|}{\textbf{Description}}                                                                                                                                                                                                            &  &  \\ \cline{1-3}
\multicolumn{1}{|l|}{0.1}              & \multicolumn{1}{l|}{22/05/2016}    & \multicolumn{1}{l|}{Vision, scope, architectural requirements and initial architecture design.}                                                                                                                                                      &  &  \\ \cline{1-3}
\multicolumn{1}{|l|}{0.2}              & \multicolumn{1}{l|}{29/07/2016}    & \multicolumn{1}{l|}{\begin{tabular}[c]{@{}l@{}}Creation of separate documents for architecture design, software\\ requirements, testing and user manual. Each populated with the relevant\\ information for the project at this stage.\end{tabular}} &  &  \\ \cline{1-3}
                                       &                                    &                                                                                                                                                                                                                                                      &  & 
\end{tabular}
\end{table}
\cleardoublepage
\setcounter{page}{1}

\textit{Disclaimer: The system is still under development. All documentation will be updated as the system progresses}

\section{System Overview}
	A World of Plants makes use of networked hardware devices to allow users to use the Internet to monitor and control the environment of living plants in real time.
	
	The system has been designed to make independent agriculture a fun and personalised experience. Compete to unlock new rewards or progress at your own leisure. Follow the plant suggestions to-the-letter or follow your own intuition. It's all up to you.
	
	Welcome to A World of Plants.

\section{System Configuration}

\section{Installation}
	\subsection{User Installation}
		The software can be found at \url{http://www.funge.cf}
		
		There is no installation necessary if you are an ordinary user, it is as simple as visiting the above listed website and registering.
		
		Registration can be done as follows:
		\begin{itemize}
			\item Visit \url{http://funge.cf}
			\item Select "Need an Account?"
			\item Enter the relevant details
			\item After selecting the "Sign Up" button you should be logged in automatically
		\end{itemize}
	
	\subsection{Software Developer Installation}
		The software can be found at \url{https://github.com/DillonHeins/Funge}
		\subsubsection{Initial Set Up}
			\begin{itemize}
				\item Set up an AWS account by visiting:
					\begin{itemize}
						\item \url{https://aws.amazon.com/}
					\end{itemize}
				\item Download and install the AWS command line interface:
					\begin{itemize}
						\item Additional information can be located at \url{https://aws.amazon.com/cli/}
					\end{itemize}
				\item Install Java Development Kit 8:
					\begin{itemize}
						\item \url{http://www.oracle.com/technetwork/java/javase/downloads/jdk8-downloads-2133151.html}
					\end{itemize}
				\item Install Apache Maven. Instructions on how to download, install and run Maven can be located at:
					\begin{itemize}
						\item \url{https://maven.apache.org/}
					\end{itemize}
			\end{itemize}
		\subsubsection{Java Backend Installation}
			\begin{itemize}
				\item Navigate to BackEnd/a-world-of-plants
				\item Open a command line window on your operating system of choice
				\item Run the command:
				\begin{itemize}
					\item 
\begin{lstlisting}
mvn test package
\end{lstlisting}
					\item This command will first test the code to ensure it is running accordingly and it will then package the software into a .jar file
				\end{itemize}
				\item Navigate to BackEnd/a-world-of-plants/target
				\item In order to upload the .jar to AWS S3 and deploy it to AWS Lambda run the following commands:
					\begin{itemize}
						\item 
\begin{lstlisting}
aws s3 cp jarname.jar s3://bucket-name
\end{lstlisting}

						\item
\begin{lstlisting}
aws lambda update-function-code --function-name function 
\--s3-bucket bucket-name --s3-key jarname.jar
\end{lstlisting}
					\end{itemize}	
			\end{itemize}
		\subsubsection{AngularJS Frontend Installation}
			\begin{itemize}
				
			\end{itemize}

\section{Getting Started}

\section{Using the System}

\section{Troubleshooting}

\section{Landing Page}
	\subsection{Logging in}
		\textcolor{red}{(1)} The first page you'll encounter is the login page. If you already have an account, enter your username and password in the appropriate blocks and click "Sign In".
		\newline
		\textcolor{red}{(2)} If you'd like to find out more about A World of Plants, you can scroll down to our information page. Scroll back up when you're ready to login or register.
		\newline
		\textcolor{red}{(3)} If you have not yet registered, click on "Need an account?" to navigate to the account registration.
		\begin{figure}[H]
			\includegraphics[width=\textwidth]{../images/UserManual/login.PNG}
			\caption{Login Page}
		\end{figure}				
		
	\subsection{Account Registration}
		\textcolor{red}{(1)} If you do not yet have an account, enter your email, the username you'd like to be known by, and your password (twice) and click the "Sign Up" button.
		\newline
		\textcolor{red}{(2)} If you already have an account and would like to log in, click on "Have an Account?" to navigate to the user login.
		\newline
		\begin{figure}[H]
			\includegraphics[width=\textwidth]{../images/UserManual/registration.PNG}
			\caption{Registration Page}
		\end{figure}
		
\section{Main Site}
	\subsection{Navbar}
	All pages on the main site will display the navbar at the top.
		\subsubsection{User Actions}
		\textcolor{red}{(1)} To access your user actions, click on your username at the right of the navbar. From here, you can either navigate to your profile or log out.
		\begin{figure}[H]
			\includegraphics[width=\textwidth]{../images/UserManual/navbar-user-actions.PNG}
			\caption{Navbar User Actions}
		\end{figure}
		
	\subsection{Plants}
	To access your plants, click on "Plants" in the sidebar. This page will display a summary of all your current plants.
	\newline
	\textcolor{red}{(1)} To add a new plant, click on the "Create New Plant" button on the top right. See \hyperref[sec:creating-a-plant]{Creating a Plant} for more information.
	\newline
	\textcolor{red}{(2)} To view more information or change some of the details for a specific plant, click on "View Details" for the plant of your choice. See \hyperref[sec:plant-details]{Plant Details} for more information.
	\begin{figure}[H]
		\includegraphics[width=\textwidth]{../images/UserManual/plants.PNG}
		\caption{Plants View}
	\end{figure}
		
		\subsubsection{Creating a Plant}
		\label{sec:creating-a-plant}
		To create a plant, enter the plant's unique details and then click on "Create".
		\begin{figure}[H]
			\includegraphics[width=\textwidth]{../images/UserManual/create-plant.PNG}
			\caption{Creating a Plant}
		\end{figure}
		
	\subsection{Plant Details}
	\label{sec:plant-details}
	From here you can get an overview of all of your plant's details.
	\newline
	\textcolor{red}{(1)} To update plant details such as the plant's name or category, make your desired changes and then click the "Update" button.
	\newline
	\textcolor{red}{(2)} If you'd like to delete your plant, click on the red "Delete" button. \textbf{This is permanent}, so be careful.
	\begin{figure}[H]
		\includegraphics[width=\textwidth]{../images/UserManual/plant-details.PNG}
		\caption{Plant Details}
	\end{figure}

\end{document}